\section{Appendix}

\subsection{SPREE} \label{sec:SPREE}

SPREE (Shared Protected Runtime Execution Enclaves) is a way for parachains to have shared code, and furthermore for the execution and state of that code to be sandboxed. From the point of view of parachain A, how much can it trust parachain B? Polkadot's shared security guarantees the correct execution of B's code with as much security as it does A's code. However, if we do not know B's code itself and even if we know the code now, maybe the governance mechanism of B can change the code and we do not trust that. This changes if we knew some of B's code, that it's governance did not have control of, and which could be sent messages by A. Then we would know how B's code would act on those messages if it was executed correctly and so shared security gives us the guarantees we need.

A SPREE module is a piece of code placed in the relay chain, that parachains can opt into. This code is part of that chains state transition validation function (STVF). The execution and state of this SPREE module are sandboxed away from the rest of the STVF's execution. SPREE modules on a remote chain can be addressed by XCMP. The distribution of messages received by a parachain would itself be controlled by a SPREE module (which would be compulsory for chains that want to use any SPREE modules).

We expect that most messages sent by XCMP will be from a SPREE module on one chain to the same SPREE module on another chain. When SPREE modules are upgraded, which involves putting updated code on the relay chain and scheduling an update block number, it is upgraded on all parachains in their next blocks. This is done in such a way as to guarantee that messages sent by a version of the SPREE module one one chain to the same module on another are never received by past versions. Thus message formats for such messages do not need to be forward compatible and we do not need standards for these formats.

For an example of the security guarantees we get from SPREE, if A has a native token, the A token, what we would like is to be sure that parachain B could not mint this token. We could enforce this by A keeping an account for B in A's state. However if an account on B want's to send some A token to a third parachain C, then it would need to inform A. A SPREE module for tokens would allow this kind of token transfer without this accounting. The module on A would just send a message to the same module on B, sending the tokens to some account. B could then send them on to C and C to A in a similar way. The module itself would account for tokens in accounts on chain B, and Polkadot's shared security and the module's code would enforce that B could never mint A tokens. XCMP's guarantee that messages will be delivered and SPREE'S guarantee that they will be interpreted correctly mean that this can be done by just sending one message per transfer and is trust free. This has applications far beyond token transfer and means that trust minimising protocols are far easier to design.

Parts of SPREEs design and implementation have yet to be fully designed. Credit goes to the reddit user u/Tawaren for the initial idea behind SPREE.

\subsection{Interoperability with External Chains}\label{sec:bridge}

Polkadot is going to host a number of bridge components to other chains. This section will be focused on bridging to BTC and ETH (1.x) and hence will mostly be reviewing bridging proof of work chains. Our bridge design is inspired by XClaim \cite{Zamyatin:2019:XClaim}.
The bridge logic will have two important parts: a bridge relay, which understands as much as possible the consensus of the bridged chain, and a bank (name change possible for PR reasons), which involves staked actors owning bridged chain tokens on behalf of Polkadot.
The bridge relay needs to be able to carry out consensus verification of the bridged chain and verify transaction inclusion proofs there. On the one hand, the bank can be used by users on the bridged chain to lock tokens as backing for the corresponding asset they want to receive on Polkadot, e.g., PolkaETH or PolkaBTC. On the other hand, users can use the bank to redeem these assets into the bridged chain tokens.
The bridge relay aims to put as much of the logic of a light/thin client of a bridged chain on a bridge parachain as is feasible – think BTC-Relay. However, crypto and storage are much cheaper on a parachain than in an ETH smart contract. We aim to put all block headers and proofs-of-inclusion of certain transactions of the bridged chain in the blocks of the bridge parachain. This is enough to decide whether a transaction is in a chain which is probably final. The idea for the bridge relay for Bitcoin and ETH1.0 is to have a longest-chain bridge chain where conflicts are resolved with a voting/attestation scheme.

\subsection{Comparison with other multi-chain systems}\label{sec:comparison}
\subsubsection{ETH2.0}
Ethereum 2.0 promises a partial transition to proof-of-stake and to deploy sharding to improve speed and throughput.  There are extensive similarities between the Polkadot and Ethereum 2.0 designs, including similar block production and finality gadgets.  

All shards in Ethereum 2.0 operate as homogeneous smart contract based chains, while parachains in Polkadot are independent heterogeneous blockchains, only some of which support different smart contract languages.  
At first blush, this simplifies deployment on Ethereum 2.0, but ``yanking'' contracts between shards dramatically complicates the Ethereum 2.0 design.  We have a smart contract language Ink! that exists so that smart contract code can more easily be migrated into being parachain code.  We assert that parachains inherent focus upon their own infrastructure should support higher performance far more easily than smart contracts.

Ethereum 2.0 asks that validators stake exactly 32 ETH, while Polkadot fixes one target number of validators, and attempts to maximise the backing stake with NPoS (see Section~\ref{sec:validators}).  At a theoretical level, we believe the 32 ETH approach results in validators being less ``independent'' than NPoS, which weakens security assumptions throughout the protocol.  We acknowledge however that Gini coefficient matters here, which gives Ethereum 2.0 an initial advantage in ``independence''.  We hope NPoS also enables more participation by Dot holders with balances below 32 ETH too.

Ethereum 2.0 has no exact analog of Polkadot's availability and validity protocol (see Section \ref{sec:validity-and-availability}).  We did however get the idea to use erasure codes from the Ethereum proposal \cite{availabilityETH2}, which aims at convincing lite clients.  
% TODO: Jeff: I dislike how this second part is written but I'm not going to rephrase it right now.
Validators in Ethereum 2.0 are assigned to each shard for attesting block of shards as parachain validators in Polkadot thus constitute the committee of the shard. The committee members receive a Merkle proof of randomly chosen code piece from a full node of the shard and verify them. If all pieces are verified and no fraud-proof is announced, then the block is considered as valid. The security of this scheme is based on having an honest majority in the committee while the security of Polkadot's scheme based on having at least one honest validator either among parachain validators or secondary checkers (see Section~\ref{sec:validity-and-availability}). Therefore, the committee size in Ethereum 2.0 is considerably large comparing to the size of parachain validators in Polkadot. 
% TODO: \cite{ByzCoin} as analogous security propertties here maybe?  Or do we talk about them elsewhere?

% TODO: Jeff: Could this last paragraph be folded into the first?
The beacon chain in Ethereum 2.0 is a proof-of-stake protocol as Polkadot's relay chain. Similarly, it has a finality gadget called Casper \cite{CasperFFG,CasperCBC} as GRANDPA in Polkadot. Casper also combines  eventual finality and  Byzantine agreement as GRANDPA but GRANDPA gives better liveness property than Casper \cite{2018:Stewart:Grandpa}.
%TODO Ask about the details of better liveness to Al


\subsubsection{Sidechains}
An alternative way to scale blockchain technologies are using side-chains \footnote{that allow tokens from one blockchain to be considered valid on an independent blockchain and be used there}. These solutions are also addressing interoperability, in that they enabling bridging side chains to the main chain. For example, for Eth1.0 many side-chains were introduced that contributed to scalability such as Plasma Cash and Loom \footnote{https://loomx.io}.
A prominent solution that is solely based on bridging independent chains to each other is Cosmos \footnote{https://cosmos.network} that is reviewed and compared to Polkadot next.

%\paragraph{ETH1.0 Sidechains, e.g., Plasma Cash}
%TODO

\subsubsection{Cosmos}

Cosmos is a system designed to solve the blockchain interoperability problem that is fundamental to improve the scalability for the decentralized web. In this sense, there are surface similarities between the two systems. Hence, Cosmos consists of components which play similar roles and resemble the sub-components of Polkadot. For example, the Cosmos Hub is used to transfer messages between Comos' zones similarly to how the Polkadot Relay Chain oversees the passing of messages among Polkadot parachains.

There are however significant differences between the two systems. Most importantly, while the Polkadot system as a whole is a sharded state machine (see Section \ref{sec:relaychain}), Cosmos does not attempt to unify the state among the zones and so the state of individual zones is not reflected in the Hub's state. As a result, a similar property as the shared security offered by Polkadot is absence in Cosmos system. Consequently, the Cosmos cross-chain messages, are no longer trust-less and the onus is on the user to trust the (validators of) sending and the receiving zones (and optionally the Hub if it is involved in the cross transaction) to engage in a particular transaction. As such the security of the user's transaction is bounded to the security least secure chain participating in that transaction. 

Similarly, the security promise of Polkadot guarantees that validated parachain data are available at a later time for retrieval and audit (see Section \ref{sec:validity-and-availability}). In contrast,  Cosmos's trust model does not guarantee any security assumption regarding the participating zones (and in particular regarding the availability of the their data). Therefore, the users are ought to trust the zone operators to keep the history of the chain state.

It is noteworthy that using the SPREE modules, Polkadot offers even stronger security than the shared security.
%(See Section \ref{sec:spree})r
When a parachain signs up for a SPREE module, Polkadot guarantees that certain XCMP messages received by that parachain are being processed by the pre-defined SPREE module set of code. No similar cross-zone trust framework is offered by the Cosmos system.

Another significant difference between Cosmos and Polkadot consists in the way the blocks are produced and finalized. In Polkadot, because all parachain states are strongly connected to relay chain states, the parachain can temporarily fork alongside the relay chain. This allows the block production to decouple from the finality logic. In this sense, the Polkadot blocks can be produced over unfinalized blocks and multiple blocks can be finalised at once. On the other hand, the Cosmos zone depends on the instant finality of the Hub's state to perform a cross-chain operation and therefore a delayed finalization halts the cross-zone operations.



\section{Glossary}



%\eray{
\begin{longtable}{p{.15\textwidth}p{.55\textwidth}p{.1\textwidth}p{.1\textwidth}} \label{t:time}
    \textbf{Name}  & \textbf{Description} & \textbf{Symbol} (plural)& \textbf{Def} \\
    \hline
    BABE & A mechanism to assign elected validators randomly to block production for a certain slot. && \ref{sec:babe} \\
    BABE Slot & A period for which a relay chain block can be produced. It's about 5 seconds. & \slot & \ref{sec:babe} \\
    Collator & Assist validators in block production. A set of collators is defined as \Col . & \col (\Col) & \ref{par:collators} \\
    Dot & The Polkadot native token. && \ref{sec:economics} \\
    Elected\newline- validators & A set of elected validators. & \Val & \\
    Epoch & A period for which randomness is generated by BABE. It's about 4 hours. & \ep & \\
    Era & A period for which a new validator set is decided. It's about 1 day. && \\
    Extrinsics & Input data supplied to the Relay Chain to transition states. && \ref{par:extrinsics} \\
    Fishermen & Monitors the network for misbehaviour. && \ref{par:fishermen} \\
    Gossiping & Broadcast every newly received message to peers. && \ref{sec:gossiping} \\
    GRANDPA & Mechanism to finalize blocks. && \ref{sec:grandpa} \\
    GRANDPA\newline- Round & A state of the GRANDPA algorithm which leads to block finalisation. && \ref{sec:grandpa} \\
    Nominator & Stake-holding party who nominates validators to be elected. A set of nominators is defined as \Nom . & \nom (\Nom) & \ref{par:nominators} \\
    NPoS & \emph{Nominated Proof-of-Stake} - Polkadot's version of PoS, where nominated validators get elected to be able to produce blocks. && \ref{sec:validators} \\
    Parachain & Heterogeneous independent chain. & \Par & \\
    PJR & \emph{Proportional-Justified-Representation} - Ensures that validators represent as many nominator minorities as possible. && \ref{sec:validators} \\
    %PoS & \emph{Proof-of-Stake} - Alternative to PoW, where parties vote with locked funds. && \ref{sec:validators} \\
    PoV & \emph{Proof-of-Validity} - Mechanism where a validator can verify a block without having its full state. && \ref{sec:parachainblockproduction} \\
    %PoW & \emph{Proof-of-Work} - Mechanism where parties vote with processing power. && \\
    Relay\newline- Chain & Ensures global consensus among parachains. && \ref{sec:relaychain} \\
    Runtime & The Wasm blob which contains the state transition functions, including other core operations required by Polkadot. && \ref{par:state_transition} \\
    Session & A session is a period of time that has a constant set of validators. Validators can only join or exit the validator set at a session change. && \\
    STVF & \emph{State-Transition-Validation-Function} - A function of the Runtime to verify the PoV. && \ref{sec:parachainblockproduction} \\
    Validator & The elected and highest in charge party who has a chance of being selected by BABE to produce a block. A set of candidate validators is defined as \Can . The number of validators to elect is defined as \nval . & \val (\Val)& \ref{par:validators} \\
    VRF & \emph{Verifiable-Random-Function} - Cryptographic function for determining elected validators for block production. && \ref{sec:babe} \\
    XCMP & A protocol that parachains use to send messages to each other. && \ref{sec:XCMP} \\
\caption{Glossary for Polkadot}
\end{longtable}
%}

%\alfonso{}{I think the table should contain more information. I would add a) possibly longer descriptions, b) a reference to the section that introduces them (and where we give an even longer description + its reason of being), and c) their lengths in seconds/minutes/hours, where we put a big note saying all lengths are tentative and subject to change considerably.}
%\alfonso{}{Also, we should either add "session" to the table, or remove all mentions of sessions. Simplifying could be a good idea, so maybe the latter?}
